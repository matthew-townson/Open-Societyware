\documentclass{report}

% Import packages
\usepackage{datetime2}
\usepackage{titling}

\righthyphenmin 62
\lefthyphenmin 62

\title{Open-Societyware \\
       Design Guide}
\author{Matthew Townson}

\begin{document}
    \begin{titlepage}
        \centering
        \vspace*{1in}
        {\Huge\bfseries\thetitle \par}
        \vspace{1.5in}
        {\Large\theauthor \par}
        \vfill
        {\large Latest revision: \DTMtoday \par}
    \end{titlepage}

    \pagenumbering{gobble}
    \newpage
    \tableofcontents
    \newpage

    \pagenumbering{roman}
    \renewcommand\thesection{\arabic{section}.0}
    \renewcommand\thesubsection{\arabic{section}.\arabic{subsection}}
    \setcounter{section}{-1}

    \raggedright

    \section{Introduction}
    \paragraph*{}
        Welcome to \textbf{Open Societyware}, the open-source, multifeatured, modular society system for everyone. \\
        The aim of this project is to provide University Societies, Clubs, and other activity groups a place to manage
        an online presence. This includes features such as keeping track of members, and their roles within the society,
        use of a website, blog, voting system, and maybe more in the future... \\
    
    \paragraph*{}
        \textbf{Open Societyware} is an open-source project, if the name wasn't enough of an indication. This means
        you are free to use the software as you like, within the constraints of the CC SA 4.0 license.

    \newpage
    \pagenumbering{arabic}
    \section{Design}



\end{document}